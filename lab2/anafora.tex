\documentclass[10pt, a4paper]{report}  % report depends on what Im writing
\usepackage[english,greek]{babel}
\usepackage[utf8]{inputenc}
\usepackage{graphicx}
\usepackage{amsmath}  % supports many of the math that I wil need
\usepackage{subfig}  % to enable many images on the same row
\usepackage[left = 25mm, top = 25mm, bottom = 25mm]{geometry}  % to define custom margins
\usepackage{hyperref}  % to reference sections and other un-referenceable stuff
\graphicspath{ {/} }
\newcommand{\en}{\selectlanguage{english}}  % define the short command we will use in the text
\newcommand{\gr}{\selectlanguage{greek}} 

\usepackage{listings}
\usepackage{xcolor}
%%%%%%%%%%%%%%%%%%%%%%%%%%%%%%%%%%%%%%%%%%%%%%%%%%%%%%%%%%% code snippet, see overleaf for healp
% Code input related settings
\definecolor{codegreen}{rgb}{0,0.6,0}
\definecolor{codegray}{rgb}{0.5,0.5,0.5}
\definecolor{codepurple}{rgb}{0.58,0,0.82}
\definecolor{backcolour}{rgb}{0.95,0.95,0.95}  % changed this color to match the plot window of Matlab

% defines the colors of the style "mystyle"
\lstdefinestyle{mystyle}{
	backgroundcolor=\color{backcolour},   
	commentstyle=\color{codegreen},
	keywordstyle=\color{magenta},
	numberstyle=\tiny\color{codegray},
	stringstyle=\color{codepurple},
	basicstyle=\ttfamily\footnotesize,
	breakatwhitespace=false,         
	breaklines=true,                 
	captionpos=b,                    
	keepspaces=true,                 
	numbers=none,     % changed this because we dont want numbered lines in 5 lines of code               
	numbersep=5pt,                  
	showspaces=false,                
	showstringspaces=false,
	showtabs=false,                  
	tabsize=2
}
% sets "mystyle" as active color profile
\lstset{style=mystyle}
%%%%%%%%%%%%%%%%%%%%%%%%%%%%%%%%%%%%%%%%%%%%%%%%%%%%%%%%%%%

\title{\textbf{$2^o$ Εργαστήριο στα Συστήματα Ελέγχου}}
\author{Ευάγγελος Κατσούπης 2017030077\\ Απόστολος Γιουμερτάκης 2017030142\\ Α. Ραφαήλ Ελληνιτάκης 2017030118\\ Κωνσταντίνος Βούλγαρης 2017030125\\ \\ \textbf{Ομάδα 37}}
\date{5 Απριλίου 2021}

\begin{document}
	
	\maketitle
	
	\chapter{Υπολογιστικό Μέρος}
	\section{Κινητήρας}
	\gr
	Έχοντας την συνάρτηση μεταφοράς που περιγράφει έναν κινητηρα, προσπαθούμε να υπολογίσουμε την συμπεριφορά του θεωρητικά σε διάφορα σενάρια φόρτου και εισόδων ελέγχου.
	Χρησιμοποιήσαμε το λογιμικό \en Matlab \gr για όλους τους υπολογισμούς και τις γραφικές παραστάσεις.
	Δεδομένης της συνάρτησης μεταφοράς που μας δίνεται:
	$$ H(s) = K_s \frac{1}{T_1s+1}\frac{1}{T_2s+1} = \frac{b_0}{s^2 + a_1s+a_0} $$
	
	\noindent
	1)Μας δίνονται οι εξής παράμετροι:
	$$K_s = 0.8, T_g=1.05sec, T_u = 0.14sec$$ 
	Υπολογίσαμε τις παραμέτρους του \en PI \gr συστήματος με την μέθοδο \en CHR overshoot 0\% \gr. Όπως και στο προηγούμενο εργαστήριο, με την μέθοδο \en pidstd() \gr ορίσαμε το σύστημα ελέγχου μας. Το \en $T_b$ \gr αντιστοιχεί στο \en $T_g$ \gr και το \en $T_e$ \gr αντιστοιχεί στο \en $T_u$ \gr όπως ξέρουμε απο θεωρία.Στον κώδικα \ref{lst:definition_of_controller_d1} βλέπουμε την υλοποίηση σε \en Matlab\gr.
	\en
	\begin{lstlisting}[language=Matlab,label={lst:definition_of_controller_d1}, caption=Definition of the PI Controller]
		Kp = (0.35*Tb)/(Ks*Te);
		Ti = 1.2*Tb;
		figure('Name','PI controller')
		PID_controller = pidstd(Kp, Ti);  % definition of the PI controller
		m = feedback(PID_controller*sys,1);  % definition of the feedback
		step(m)  % step response of the system
	\end{lstlisting}
	\gr

	\begin{figure}[h] 
		\centering
		\includegraphics[width=0.5\textwidth]{d1_graph}
		\caption{Η απόκριση του νέου συστήματος σε ένα \en step \gr απο τα 5 στα 7 \en volts\gr.}
		\label{fig:d1_graph}
	\end{figure}
	
	\newpage

	
	2) Για το δεύτερο ερώτημα, πρέπει να δημιουργήσουμε τον παλμό που εικονίζεται στην εκφώνηση, με πλάτος \en peak to peak \gr 500$\frac{n_w}{rpm}$, θετικό \en offset \gr 1500, και περίοδο 16 δευτερόλεπτα. Το κάνουμε όπως φαίνεται στον κώδικα παρακάτω, στοn κώδικα \ref{lst:square_pulse_definition}. 
	\en
	\begin{lstlisting}[language=Matlab,label={lst:square_pulse_definition}, caption=Definition of the square pulse]
		f = 1/16;  % definition of the frequency
		t1=0:0.01:60;  
		pulse = 500*square(2*pi*f*t1)/2 + 1750;  % definition of the square pulse
		plot(t1, pulse)
		axis([0 60 0 4000])
		hold on  
		resp = lsim(m,pulse,t1);  % feeding the pulse into the control system m
		plot(t1, resp) 
		e = resp - (pulse.');  % defining the error ,to use in the error indices later
	\end{lstlisting}
	\gr


	Ο παλμός που δημιουργήθηκε, δόθηκε ώς είσοδος στο σύστημα που δημιουργήσαμε στο προηγούμενο ερώτημα και η απόκριση του φαίνεται στο σχήμα \ref{fig:d2_pulse_graph}.
	
	
	\begin{figure}[h] 
		\centering
		\includegraphics[width=0.5\textwidth]{d2_pulse_graph}
		\caption{Εδώ φαίνεται η απόκριση του συστήματος ελέγχου στο σήμα \en PWM \gr με \en CHR \gr μέθοδο.}
		\label{fig:d2_pulse_graph}
	\end{figure} 


	Για να βρούμε τους δείκτες σφάλματος \en ISE, IAE, ITAE \gr και \en ITSE \gr, πρέπει πρώτα να ορίσουμε το σφάλμα:
	$$e(t) = |response(t)-pulse(t)|$$
	Το οποίο υλοποιήθηκε με ώς \en \textbf{e = resp - (pulse.');} \gr(με όρους \en Matlab\gr) με την τελεία και την απόστροφο να δείχνουν τον ανάστροφο πίνακα του \en pulse \gr για να μπορεί να γίνει η πράξη με την απόκρισγ.
	Έπειτα, το υποβάλλουμε σε διάφορες ολοκληρώσεις, όπως:
	$$ISE = \int_{0}^{\infty} e^2(t) \,dx \text{,     } ISE = \int_{0}^{\infty} |e(t)| \,dx  $$
	$$ITSE = \int_{0}^{\infty} te^2(t) \,dx \text{,     }  ITAE = \int_{0}^{\infty} t|e(t)| \,dx  $$
	Τα παραπάνω υλοποιήθηκαν με την συνάρτηση \en trapz() \gr της \en Matlab \gr, που υλοποιεί ολοκλήρωση χωρίζοντας το επίπεδο σε τραπέζια, όπως φαίνεται στην εικόνα \ref{fig:d2_errors}
	\en
	\begin{lstlisting}[language=Matlab, caption=Definition of the Error Indices]
		iea = trapz(t1,abs(e));          % IAE trapz=numerical integration 
		ise = trapz(t1,e.^2);            % ISE 
		itae = trapz(t1, t1'.*abs(e));     % ITAE
		itse = trapz(t1,t1'.*(e.^2));      % ITSE
	\end{lstlisting}
	\gr 

	\newpage
	Με τα παραπάνω, καταλήξαμε στις εξής τιμές δεικτών σφαλμάτων:
	$$IAE = 2.737$$
	$$ISE = 1.52$$
	$$ITAE = 5.79$$
	$$ITSE = 1.53$$	
	
	3)Ρύθμιση με την εμπειρική μέθοδο $T_{sum}$
	Για να ρυθμίσουμε τον ελεγκτή με την συγκεκριμένη μέθοδο, αρκεί να συμβουλευθούμε τον πίνακα που ξέρουμε για την ρύθμιση $T_{sum}$, όπως φαίνεται παρακάτω:
	\begin{center}
		\begin{tabular}{|c|c|c|c|}
			\hline
			Ελεγκτής & $K$ & $T_i$ & $T_d$ \\
			\hline
			$PI$ & $\frac{0.5}{K_s}$ & $0.5T_S$ & - \\
			\hline
		\end{tabular}
	\end{center}


	Και στο παρακάτω πεδίο δίνεται ο κώδικας για την υλοποίηση. Γνωρίζουμε οτι $T_s = T_1 + T_2$ απο θεωρία.:
	\en
	\begin{lstlisting}[language=Matlab, caption= \gr Νέος ορισμός $K_p$ και $T_i$ για το νέο εμπειρικά ρυθμισμένο σύστημα.\en]
		Kp = 0.5/Ks;  % Tsum Method to find Kp
		Ti = 0.5*(T1+T2);  % Tsum Method to find Ti
		PID_controller = pidstd(Kp, Ti);  % definition of the PI system
		m = feedback(PID_controller*sys,1);  % the feedback of the system
	\end{lstlisting}
	\gr 


	Στην εικόνα \ref{fig:d3_pulse_graph} φαίνεται η απόκριση στον \en "PWM" \gr παλμό του κινητήρα. Είναι εμφανώς ομαλότερη απο την μέθοδο \en CHR 0\% overshoot \gr, αλλα με μεγαλύτερο χρόνο απόκρισης.
	\begin{figure}[h] 
		\centering
		\includegraphics[width=0.5\textwidth]{d3_pulse_graph}
		\caption{Η απόκριση του συστήματος που είναι ρυθμισμένο με την μέθοδο $T_sum$. }
		\label{fig:d3_pulse_graph}
	\end{figure} 
	
		
\end{document}


