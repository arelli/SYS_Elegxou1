\documentclass[10pt, a4paper]{report}  % report depends on what Im writing
\usepackage[english,greek]{babel}
\usepackage[utf8]{inputenc}
\usepackage{graphicx}
\usepackage{amsmath}  % supports many of the math that I wil need
\usepackage{subfig}  % to enable many images on the same row
\usepackage[left = 25mm, top = 25mm, bottom = 25mm]{geometry}  % to define custom margins
\usepackage{hyperref}  % to reference sections and other un-referenceable stuff
\graphicspath{ {/} }
\newcommand{\en}{\selectlanguage{english}}  % define the short command we will use in the text
\newcommand{\gr}{\selectlanguage{greek}} 

\title{\textbf{$2^o$ Εργαστήριο στα Συστήματα Ελέγχου}}
\author{Ευάγγελος Κατσούπης 2017030077\\ Απόστολος Γιουμερτάκης 2017030142\\ Α. Ραφαήλ Ελληνιτάκης 2017030118\\ Κωνσταντίνος Βούλγαρης 2017030125\\ \\ \textbf{Ομάδα 37}}
\date{5 Απριλίου 2021}

\begin{document}
	
	\maketitle
	
	\chapter{Α' Μέρος Άσκησης}
	\section{Υπολογιστικό Μέρος}
	\gr
	Χρησιμοποιήσαμε το λογιμικό Matlab για όλους τους υπολογισμούς και τις γραφικές παραστάσεις.
	Δεδομένης της συνάρτησης μεταφοράς που μας δίνεται:
	$$ H(s) = K_s \frac{1}{T_1s+1}\frac{1}{T_2s+1} = \frac{b_0}{s^2 + a_1s+a_0} $$
	
	
	1)Μας δίνονται οι εξής παράμετροι:
	$$K_s = 0.8, T_g=1.05sec, T_u = 0.14sec$$ 
	Υπολογίσαμε τις παραμέτρους του \en PI \gr συστήματος με την μέθοδο \en CHR overshoot 0\% \gr όπως φαίνεται στην εικόνα \ref{fig:d1}. Όπως και στο προηγούμενο εργαστήριο, με την μέθοδο \en pidstd() \gr ορίσαμε το σύστημα ελέγχου μας. Το \en $T_b$ \gr αντιστοιχεί στο \en $T_g$ \gr και το \en $T_e$ \gr αντιστοιχεί στο \en $T_u$ \gr όπως ξέρουμε απο θεωρία.
	\begin{figure}[h] 
		\centering
		\includegraphics[width=0.5\textwidth]{d1}
		\caption{Ο ορισμός του \en PI \gr συστήματος και ο υπολογισμός παραμέτρων του.}
		\label{fig:d1}
	\end{figure} 
	
	
	2) Για το δεύτερο ερώτημα, πρέπει να δημιουργήσουμε τον παλμό που εικονίζεται στην εκφώνηση, με πλάτος \en peak to peak \gr 500$\frac{n_w}{rpm}$ και περίοδο 15 δευτερόλεπτα. Το κάνουμε όπως φαίνεται στην εικόνα
	
\end{document}


