\documentclass[10pt, a4paper]{report}  % report depends on what Im writing
\usepackage[english,greek]{babel}
\usepackage[utf8]{inputenc}
\usepackage{graphicx}
\usepackage{amsmath}  % supports many of the math that I wil need
\usepackage{subfig}  % to enable many images on the same row
\usepackage[left = 25mm, top = 25mm, bottom = 25mm]{geometry}  % to define custom margins
\usepackage{hyperref}  % to reference sections and other un-referenceable stuff
\graphicspath{ {/} }
\newcommand{\en}{\selectlanguage{english}}  % define the short command we will use in the text
\newcommand{\gr}{\selectlanguage{greek}} 

\title{\textbf{$2^o$ Εργαστήριο στα Συστήματα Ελέγχου}}
\author{Ευάγγελος Κατσούπης 2017030077\\ Απόστολος Γιουμερτάκης 2017030142\\ Α. Ραφαήλ Ελληνιτάκης 2017030118\\ Κωνσταντίνος Βούλγαρης 2017030125\\ \\ \textbf{Ομάδα 37}}
\date{5 Απριλίου 2021}

\begin{document}
	
	\maketitle
	
	\chapter{Υπολογιστικό Μέρος}
	\section{Κινητήρας}
	\gr
	Χρησιμοποιήσαμε το λογιμικό Matlab για όλους τους υπολογισμούς και τις γραφικές παραστάσεις.
	Δεδομένης της συνάρτησης μεταφοράς που μας δίνεται:
	$$ H(s) = K_s \frac{1}{T_1s+1}\frac{1}{T_2s+1} = \frac{b_0}{s^2 + a_1s+a_0} $$
	
	
	1)Μας δίνονται οι εξής παράμετροι:
	$$K_s = 0.8, T_g=1.05sec, T_u = 0.14sec$$ 
	Υπολογίσαμε τις παραμέτρους του \en PI \gr συστήματος με την μέθοδο \en CHR overshoot 0\% \gr όπως φαίνεται στην εικόνα \ref{fig:d1}. Όπως και στο προηγούμενο εργαστήριο, με την μέθοδο \en pidstd() \gr ορίσαμε το σύστημα ελέγχου μας. Το \en $T_b$ \gr αντιστοιχεί στο \en $T_g$ \gr και το \en $T_e$ \gr αντιστοιχεί στο \en $T_u$ \gr όπως ξέρουμε απο θεωρία.
	\begin{figure}[h] 
		\centering
		\includegraphics[width=0.4\textwidth]{d1}
		\caption{Ο ορισμός του \en PI \gr συστήματος και ο υπολογισμός παραμέτρων του.}
		\label{fig:d1}
	\end{figure} 

	\begin{figure}[h] 
		\centering
		\includegraphics[width=0.5\textwidth]{d1_graph}
		\caption{Η απόκριση του νέου συστήματος.}
		\label{fig:d1_graph}
	\end{figure}
	
	\newpage

	
	2) Για το δεύτερο ερώτημα, πρέπει να δημιουργήσουμε τον παλμό που εικονίζεται στην εκφώνηση, με πλάτος \en peak to peak \gr 500$\frac{n_w}{rpm}$, θετικό \en offset \gr 1500, και περίοδο 16 δευτερόλεπτα. Το κάνουμε όπως φαίνεται στην εικόνα \ref{fig:d2_pulse}
	\begin{figure}[h] 
		\centering
		\includegraphics[width=0.5\textwidth]{d2_pulse}
		\caption{Η δημιουργία του παλμού με όρους \en Matlab \gr.}
		\label{fig:d2_pulse}
	\end{figure} 
	Ο παλμός που δημιουργήθηκε, δόθηκε ώς είσοδος στο σύστημα που δημιουργήσαμε στο προηγούμενο ερώτημα και η απόκριση του φαίνεται στο σχήμα \ref{fig:d2_pulse_graph}.
	\begin{figure}[h] 
		\centering
		\includegraphics[width=0.7\textwidth]{d2_pulse_graph}
		\caption{Εδώ φαίνεται η απόκριση του συστήματος ελέγχου στο σήμα που είναι ουσιαστικά \en PWM \gr.}
		\label{fig:d2_pulse_graph}
	\end{figure} 


	Για να βρούμε τους δείκτες σφάλματος \en ISE, IAE, ITAE \gr και \en ITSE \gr, πρέπει πρώτα να ορίσουμε το σφάλμα:
	$$e(t) = |response(t)-pulse(t)|$$
	Το οποίο υλοποιήθηκε με την εντολή \en \textbf{e = resp - (pulse.');} \gr με την τελεία και την απόστροφο να δείχνουν τον ανάστροφο πίνακα του \en pulse \gr για να μπορεί να γίνει η πράξη με την απόκρισι.
	Έπειτα, το υποβάλλουμε σε διάφορες ολοκληρώσεις, όπως:
	$$ISE = \int_{0}^{\infty} e^2(t) \,dx  $$
	$$ISE = \int_{0}^{\infty} |e(t)| \,dx  $$
	$$ITSE = \int_{0}^{\infty} te^2(t) \,dx  $$
	$$ITAE = \int_{0}^{\infty} t|e(t)| \,dx  $$
	Τα παραπάνω υλοποιήθηκαν με την συνάρτηση \en trapz() \gr της \en Matlab \gr, που υλοποιεί ολοκλήρωση χωρίζοντας το επίπεδο σε τραπέζια, όπως φαίνεται στην εικόνα \ref{fig:d2_errors}
	\begin{figure}[h] 
		\centering
		\includegraphics[width=0.5\textwidth]{d2_errors}
		\caption{Η υλοποίηση των δεικτών σφάλματος.}
		\label{fig:d2_errors}
	\end{figure} 


	Με τα παραπάνω, καταλήξαμε στις εξής τιμές δεικτών σφαλμάτων:
	$$IAE = 2.737$$
	$$ISE = 1.52$$
	$$ITAE = 5.79$$
	$$ITSE = 1.53$$	
	
	3)Ρύθμιση με την εμπειρική μέθοδο $T_{sum}$
	Για να ρυθμίσουμε τον ελεγκτή με την συγκεκριμένη μέθοδο, αρκεί να συμβουλευθούμε τον πίνακα \ref{fig:tsum_method}.
	\begin{figure}[h] 
		\centering
		\includegraphics[width=0.7\textwidth]{tsum_method}
		\caption{Ο πίνακας εμπειρικής ρύθμισης $T_{sum}$.}
		\label{fig:tsum_method}
	\end{figure} 


	Και στην εικόνα \ref{fig:tsum_code} δίνεται ο κώδικας για την υλοποίηση. Γνωρίζουμε οτι $T_s = T_1 + T_2$ απο θεωρία.:
	\begin{figure}[h] 
		\centering
		\includegraphics[width=0.3\textwidth]{tsum_code}
		\caption{Νέος ορισμός $K_p$ και $T_i$ για το νέο εμπειρικά ρυθμισμένο σύστημα. }
		\label{fig:tsum_code}
	\end{figure} 
	
	
		
\end{document}


